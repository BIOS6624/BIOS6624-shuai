% Options for packages loaded elsewhere
\PassOptionsToPackage{unicode}{hyperref}
\PassOptionsToPackage{hyphens}{url}
%
\documentclass[
  12pt,
]{article}
\usepackage{amsmath,amssymb}
\usepackage{iftex}
\ifPDFTeX
  \usepackage[T1]{fontenc}
  \usepackage[utf8]{inputenc}
  \usepackage{textcomp} % provide euro and other symbols
\else % if luatex or xetex
  \usepackage{unicode-math} % this also loads fontspec
  \defaultfontfeatures{Scale=MatchLowercase}
  \defaultfontfeatures[\rmfamily]{Ligatures=TeX,Scale=1}
\fi
\usepackage{lmodern}
\ifPDFTeX\else
  % xetex/luatex font selection
\fi
% Use upquote if available, for straight quotes in verbatim environments
\IfFileExists{upquote.sty}{\usepackage{upquote}}{}
\IfFileExists{microtype.sty}{% use microtype if available
  \usepackage[]{microtype}
  \UseMicrotypeSet[protrusion]{basicmath} % disable protrusion for tt fonts
}{}
\makeatletter
\@ifundefined{KOMAClassName}{% if non-KOMA class
  \IfFileExists{parskip.sty}{%
    \usepackage{parskip}
  }{% else
    \setlength{\parindent}{0pt}
    \setlength{\parskip}{6pt plus 2pt minus 1pt}}
}{% if KOMA class
  \KOMAoptions{parskip=half}}
\makeatother
\usepackage{xcolor}
\usepackage[margin=1in]{geometry}
\usepackage{graphicx}
\makeatletter
\def\maxwidth{\ifdim\Gin@nat@width>\linewidth\linewidth\else\Gin@nat@width\fi}
\def\maxheight{\ifdim\Gin@nat@height>\textheight\textheight\else\Gin@nat@height\fi}
\makeatother
% Scale images if necessary, so that they will not overflow the page
% margins by default, and it is still possible to overwrite the defaults
% using explicit options in \includegraphics[width, height, ...]{}
\setkeys{Gin}{width=\maxwidth,height=\maxheight,keepaspectratio}
% Set default figure placement to htbp
\makeatletter
\def\fps@figure{htbp}
\makeatother
\setlength{\emergencystretch}{3em} % prevent overfull lines
\providecommand{\tightlist}{%
  \setlength{\itemsep}{0pt}\setlength{\parskip}{0pt}}
\setcounter{secnumdepth}{5}
\usepackage{setspace}
\onehalfspacing
\usepackage{helvet}
\renewcommand{\familydefault}{\sfdefault}
\ifLuaTeX
  \usepackage{selnolig}  % disable illegal ligatures
\fi
\IfFileExists{bookmark.sty}{\usepackage{bookmark}}{\usepackage{hyperref}}
\IfFileExists{xurl.sty}{\usepackage{xurl}}{} % add URL line breaks if available
\urlstyle{same}
\hypersetup{
  pdftitle={Project 1 data analysis plan},
  pdfauthor={Shuai Zhu},
  hidelinks,
  pdfcreator={LaTeX via pandoc}}

\title{Project 1 data analysis plan}
\author{Shuai Zhu}
\date{2024-09-23}

\begin{document}
\maketitle

\hypertarget{introduction}{%
\section{Introduction}\label{introduction}}

The secondary data analysis of the Multicenter AIDS Cohort Study, a
prospective cohort study designed to understand the natural and treated
histories of HIV-1 infection in homosexual and bisexual men in 4 major
in the United States. The dataset used in this analysis includes eight
years of longitudinal data from 715 HIV-infected men, capturing
laboratory measurements, quality of life scores, demographic
information, and other health-related data collected after initiating
highly active antiretroviral therapy (HAART). Multicenter AIDS Cohort
Study is a prospective cohort study designed to understand the natural
and treated histories of HIV-1 infection in homosexual and bisexual men
in 4 major in the United States. The main research question is to know
how treatment response two years after initiating HAART differs between
individuals who reported using hard drugs at baseline and those who did
not.

\hypertarget{method}{%
\section{Method}\label{method}}

The dataset was filtered to include only baseline and year 2
observations. The data was reshaped from long to wide format for ease of
analysis. A new binary variable, adh, was created, where adh = 1 for
participants with an adherence (ADH) score of 3 or less, and adh = 0
otherwise. The outcome variables include viral load, CD4+ T cell count,
physical quality of life score, and mental quality of life score.
Potential predictor variables, such as age, hard drug use, BMI,
education status, Adherence, and baseline measurements, were considered
for univariable analysis. Predictor variables that showed a significant
association with the outcomes were included in multivariable regression
models for each of the four outcome measures. The assumptions of
linearity, homoscedasticity, and normality will be assessed using
diagnostic plots. P values \textless0.05 will be considered significant.
R version 4.3.3 was used for data cleaning and analysis.

\hypertarget{preliminary-results}{%
\section{Preliminary Results}\label{preliminary-results}}

In the univariable regression analysis, several predictor variables
demonstrated significant associations with the outcomes:

\begin{itemize}
\tightlist
\item
  Viral Load: Education status, adherence, and baseline viral load were
  significantly associated with viral load at two years.
\item
  CD4+ T Cell Count: Significant predictors for CD4+ T cell count
  included education status and baseline CD4+ T cell count.
\item
  Physical Quality of Life Score: Age, education status, baseline
  physical quality of life score, and hard drug use were significantly
  associated with the physical quality of life score.
\item
  Mental Quality of Life Score: Age, BMI, education status, and baseline
  mental quality of life score showed significant associations with the
  mental quality of life score.
\end{itemize}

\end{document}
