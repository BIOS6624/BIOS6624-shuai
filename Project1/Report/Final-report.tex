% Options for packages loaded elsewhere
\PassOptionsToPackage{unicode}{hyperref}
\PassOptionsToPackage{hyphens}{url}
%
\documentclass[
  12pt,
]{article}
\usepackage{amsmath,amssymb}
\usepackage{iftex}
\ifPDFTeX
  \usepackage[T1]{fontenc}
  \usepackage[utf8]{inputenc}
  \usepackage{textcomp} % provide euro and other symbols
\else % if luatex or xetex
  \usepackage{unicode-math} % this also loads fontspec
  \defaultfontfeatures{Scale=MatchLowercase}
  \defaultfontfeatures[\rmfamily]{Ligatures=TeX,Scale=1}
\fi
\usepackage{lmodern}
\ifPDFTeX\else
  % xetex/luatex font selection
\fi
% Use upquote if available, for straight quotes in verbatim environments
\IfFileExists{upquote.sty}{\usepackage{upquote}}{}
\IfFileExists{microtype.sty}{% use microtype if available
  \usepackage[]{microtype}
  \UseMicrotypeSet[protrusion]{basicmath} % disable protrusion for tt fonts
}{}
\makeatletter
\@ifundefined{KOMAClassName}{% if non-KOMA class
  \IfFileExists{parskip.sty}{%
    \usepackage{parskip}
  }{% else
    \setlength{\parindent}{0pt}
    \setlength{\parskip}{6pt plus 2pt minus 1pt}}
}{% if KOMA class
  \KOMAoptions{parskip=half}}
\makeatother
\usepackage{xcolor}
\usepackage[margin=1in]{geometry}
\usepackage{graphicx}
\makeatletter
\def\maxwidth{\ifdim\Gin@nat@width>\linewidth\linewidth\else\Gin@nat@width\fi}
\def\maxheight{\ifdim\Gin@nat@height>\textheight\textheight\else\Gin@nat@height\fi}
\makeatother
% Scale images if necessary, so that they will not overflow the page
% margins by default, and it is still possible to overwrite the defaults
% using explicit options in \includegraphics[width, height, ...]{}
\setkeys{Gin}{width=\maxwidth,height=\maxheight,keepaspectratio}
% Set default figure placement to htbp
\makeatletter
\def\fps@figure{htbp}
\makeatother
\setlength{\emergencystretch}{3em} % prevent overfull lines
\providecommand{\tightlist}{%
  \setlength{\itemsep}{0pt}\setlength{\parskip}{0pt}}
\setcounter{secnumdepth}{5}
\usepackage{setspace}
\onehalfspacing
\usepackage{booktabs}
\usepackage{longtable}
\usepackage{array}
\usepackage{multirow}
\usepackage{wrapfig}
\usepackage{float}
\usepackage{colortbl}
\usepackage{pdflscape}
\usepackage{tabu}
\usepackage{threeparttable}
\usepackage{threeparttablex}
\usepackage[normalem]{ulem}
\usepackage{makecell}
\usepackage{xcolor}
\ifLuaTeX
  \usepackage{selnolig}  % disable illegal ligatures
\fi
\IfFileExists{bookmark.sty}{\usepackage{bookmark}}{\usepackage{hyperref}}
\IfFileExists{xurl.sty}{\usepackage{xurl}}{} % add URL line breaks if available
\urlstyle{same}
\hypersetup{
  pdftitle={Project 1 final report},
  pdfauthor={Shuai Zhu},
  hidelinks,
  pdfcreator={LaTeX via pandoc}}

\title{Project 1 final report}
\author{Shuai Zhu}
\date{2024-10-04}

\begin{document}
\maketitle

\hypertarget{introduction}{%
\section{Introduction}\label{introduction}}

The data used in this analysis come from the ongoing Multicenter AIDS
Cohort Study (MACS), a prospective cohort study designed to understand
the natural and treated histories of HIV-1 infection in homosexual and
bisexual men across four major cities in the United States. This dataset
includes eight years of longitudinal data from 715 HIV-infected men,
capturing laboratory measurements, quality of life scores, demographic
information, and other health-related data collected after the
initiation of highly active antiretroviral therapy (HAART), which is the
standard treatment for patients with HIV.

The primary research question is to examine how treatment response, two
years after initiating HAART, differs between individuals who reported
hard drug use at baseline and those who did not. Four key measures of
treatment response are considered: viral load, CD4+ T cell counts, and
physical and mental quality of life scores.

\hypertarget{methods}{%
\section{Methods}\label{methods}}

\hypertarget{data-cleaning}{%
\subsection{Data cleaning}\label{data-cleaning}}

Baseline and two year measurement was filtered for further analysis
since the purpose of this project is find out how treatment response
differ two years after treatment. The data with BMI greater than 200 or
less than 0 was removed since it is impossible. The records with
complete case was used for further analysis and the number of
observations was reduced to 425. Furthermore, BMI was categorized to
four levels underweight (BMI \textless{} 18.5 kg/m2), healthy (BMI 18.5
- 24.9 kg/m2), overweight(BMI 24.9 - 30 kg/m2) and obese (BMI
\textgreater{} 30 kg/m2). Adherence was dichotomized into
\textgreater=95\% and \textless95\%. The education levels was collapsed
into three levels (High school or before, Some college, and Graduate or
Post-graduate).

\hypertarget{data-analysis}{%
\subsection{Data analysis}\label{data-analysis}}

Both frequentist and Bayesian approaches were employed to assess
differences in treatment response by baseline hard drug use. The four
key outcomes used to assess treatment response were viral load, CD4+ T
cell counts, and physical and mental quality of life scores (AGG\_PHYS
and AGG\_MENT). The objective was to model the impact of hard drug use,
adjusting for several covariates, including baseline treatment response,
BMI, age, education level, and adherence. Viral load, due to its skewed
distribution, was log-transformed to meet the assumption of normality.
For the frequentist approach, four multivariable linear regression
models were fitted, each predicting one of the treatment response
outcomes. The model assumptions---independence, linearity,
homoscedasticity, and normality---were carefully evaluated using
standard diagnostic tools. Independence was verified by inspecting the
structure of the data and residuals. Linearity was checked by assessing
the relationship between predictors and outcome variables using residual
plots. Homoscedasticity was evaluated using residuals vs.~fitted values
plots. Normality of residuals was tested through QQ-plots.

For Bayesian regression models, both non-informative and vague priors
were used. The non-informative priors of beta are distributed with mean
0 and standard deviation 10\^{}7. The vague priors of beta are
distributed with mean 0 and standard deviation 10\^{}6. The prior
distribution for the model error was set as a half-Cauchy distribution
with a scale parameter of 2.5. Bayesian inference was carried out using
Markov Chain Monte Carlo (MCMC) sampling. Each model was run with 4 MCMC
chains, with each chain consisting of 2,000 iterations, including a
1,000 iteration burn-in period to ensure convergence. The posterior
distributions for all model parameters were summarized, and credible
intervals were used to quantify uncertainty in the estimates.

For the frequentist models, standard metrics such as p-values and
confidence intervals were used to assess the significance and effect
sizes of the predictors. For the Bayesian models, the convergence of
MCMC chains was assessed through trace plots. Posterior means and 95\%
credible intervals were reported for each parameter to provide a full
picture of the uncertainty around the estimates.

\hypertarget{result}{%
\section{Result}\label{result}}

\begin{table}

\caption{\label{tab:unnamed-chunk-2}Summary of outcomes and predictors by hard drugs}
\centering
\fontsize{10}{12}\selectfont
\begin{tabular}[t]{llll}
\toprule
  & 0 & 1 & Overall\\
\midrule
 & (N=390) & (N=35) & (N=425)\\
\addlinespace[0.3em]
\multicolumn{4}{l}{\textbf{Log-transformed Viral Load at Baseline}}\\
\hspace{1em}Mean (SD) & 10.5 (2.05) & 10.6 (2.01) & 10.5 (2.05)\\
\hspace{1em}Median [Min, Max] & 10.4 [2.20, 19.1] & 10.3 [6.61, 14.7] & 10.4 [2.20, 19.1]\\
\addlinespace[0.3em]
\multicolumn{4}{l}{\textbf{Log-transformed Viral Load at Year 2}}\\
\hspace{1em}Mean (SD) & 4.07 (2.73) & 4.13 (3.30) & 4.07 (2.78)\\
\hspace{1em}Median [Min, Max] & 3.43 [-1.40, 13.5] & 3.74 [-0.301, 13.0] & 3.43 [-1.40, 13.5]\\
\addlinespace[0.3em]
\multicolumn{4}{l}{\textbf{CD4+ T Cell Count at Baseline}}\\
\hspace{1em}Mean (SD) & 377 (197) & 361 (204) & 376 (197)\\
\hspace{1em}Median [Min, Max] & 361 [12.4, 1220] & 453 [10.9, 650] & 361 [10.9, 1220]\\
\addlinespace[0.3em]
\multicolumn{4}{l}{\textbf{CD4+ T Cell Count at Year 2}}\\
\hspace{1em}Mean (SD) & 565 (258) & 372 (252) & 549 (263)\\
\hspace{1em}Median [Min, Max] & 544 [39.5, 1730] & 357 [60.0, 971] & 529 [39.5, 1730]\\
\addlinespace[0.3em]
\multicolumn{4}{l}{\textbf{Physical Quality of Life at Baseline}}\\
\hspace{1em}Mean (SD) & 51.5 (8.67) & 48.8 (6.86) & 51.3 (8.56)\\
\hspace{1em}Median [Min, Max] & 53.7 [22.4, 69.0] & 46.7 [31.4, 62.9] & 53.5 [22.4, 69.0]\\
\addlinespace[0.3em]
\multicolumn{4}{l}{\textbf{Physical Quality of Life at Year 2}}\\
\hspace{1em}Mean (SD) & 50.0 (9.94) & 44.4 (12.1) & 49.5 (10.2)\\
\hspace{1em}Median [Min, Max] & 53.3 [14.8, 68.9] & 45.5 [18.2, 63.9] & 53.2 [14.8, 68.9]\\
\addlinespace[0.3em]
\multicolumn{4}{l}{\textbf{Mental Quality of Life at Baseline}}\\
\hspace{1em}Mean (SD) & 44.9 (13.9) & 42.6 (11.3) & 44.7 (13.7)\\
\hspace{1em}Median [Min, Max] & 49.3 [7.23, 66.0] & 45.1 [22.9, 59.6] & 48.9 [7.23, 66.0]\\
\addlinespace[0.3em]
\multicolumn{4}{l}{\textbf{Mental Quality of Life at Year 2}}\\
\hspace{1em}Mean (SD) & 47.7 (11.6) & 46.2 (14.2) & 47.6 (11.8)\\
\hspace{1em}Median [Min, Max] & 51.2 [10.5, 66.7] & 49.6 [21.3, 65.3] & 51.2 [10.5, 66.7]\\
\addlinespace[0.3em]
\multicolumn{4}{l}{\textbf{Age (years)}}\\
\hspace{1em}Mean (SD) & 43.0 (8.82) & 44.2 (9.43) & 43.1 (8.86)\\
\hspace{1em}Median [Min, Max] & 43.0 [20.0, 73.0] & 47.0 [29.0, 61.0] & 43.0 [20.0, 73.0]\\
\addlinespace[0.3em]
\multicolumn{4}{l}{\textbf{Body Mass Index (kg/m²)}}\\
\hspace{1em}Healthy & 192 (49.2\%) & 25 (71.4\%) & 217 (51.1\%)\\
\hspace{1em}Obsese & 47 (12.1\%) & 2 (5.7\%) & 49 (11.5\%)\\
\hspace{1em}Overweight & 138 (35.4\%) & 7 (20.0\%) & 145 (34.1\%)\\
\hspace{1em}Underweight & 13 (3.3\%) & 1 (2.9\%) & 14 (3.3\%)\\
\addlinespace[0.3em]
\multicolumn{4}{l}{\textbf{Adherence Level}}\\
\hspace{1em}Mean (SD) & 0.897 (0.304) & 1.00 (0) & 0.906 (0.292)\\
\hspace{1em}Median [Min, Max] & 1.00 [0, 1.00] & 1.00 [1.00, 1.00] & 1.00 [0, 1.00]\\
\addlinespace[0.3em]
\multicolumn{4}{l}{\textbf{Education Level}}\\
\hspace{1em}Graduate, Post Graduate & 81 (20.8\%) & 9 (25.7\%) & 90 (21.2\%)\\
\hspace{1em}High school & 80 (20.5\%) & 12 (34.3\%) & 92 (21.6\%)\\
\hspace{1em}some college & 229 (58.7\%) & 14 (40.0\%) & 243 (57.2\%)\\
\bottomrule
\end{tabular}
\end{table}

\hypertarget{conclusion}{%
\section{Conclusion}\label{conclusion}}

\end{document}
