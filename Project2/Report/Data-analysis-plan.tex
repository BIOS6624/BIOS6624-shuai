% Options for packages loaded elsewhere
\PassOptionsToPackage{unicode}{hyperref}
\PassOptionsToPackage{hyphens}{url}
%
\documentclass[
  12pt,
]{article}
\usepackage{amsmath,amssymb}
\usepackage{iftex}
\ifPDFTeX
  \usepackage[T1]{fontenc}
  \usepackage[utf8]{inputenc}
  \usepackage{textcomp} % provide euro and other symbols
\else % if luatex or xetex
  \usepackage{unicode-math} % this also loads fontspec
  \defaultfontfeatures{Scale=MatchLowercase}
  \defaultfontfeatures[\rmfamily]{Ligatures=TeX,Scale=1}
\fi
\usepackage{lmodern}
\ifPDFTeX\else
  % xetex/luatex font selection
\fi
% Use upquote if available, for straight quotes in verbatim environments
\IfFileExists{upquote.sty}{\usepackage{upquote}}{}
\IfFileExists{microtype.sty}{% use microtype if available
  \usepackage[]{microtype}
  \UseMicrotypeSet[protrusion]{basicmath} % disable protrusion for tt fonts
}{}
\makeatletter
\@ifundefined{KOMAClassName}{% if non-KOMA class
  \IfFileExists{parskip.sty}{%
    \usepackage{parskip}
  }{% else
    \setlength{\parindent}{0pt}
    \setlength{\parskip}{6pt plus 2pt minus 1pt}}
}{% if KOMA class
  \KOMAoptions{parskip=half}}
\makeatother
\usepackage{xcolor}
\usepackage[margin=1in]{geometry}
\usepackage{graphicx}
\makeatletter
\def\maxwidth{\ifdim\Gin@nat@width>\linewidth\linewidth\else\Gin@nat@width\fi}
\def\maxheight{\ifdim\Gin@nat@height>\textheight\textheight\else\Gin@nat@height\fi}
\makeatother
% Scale images if necessary, so that they will not overflow the page
% margins by default, and it is still possible to overwrite the defaults
% using explicit options in \includegraphics[width, height, ...]{}
\setkeys{Gin}{width=\maxwidth,height=\maxheight,keepaspectratio}
% Set default figure placement to htbp
\makeatletter
\def\fps@figure{htbp}
\makeatother
\setlength{\emergencystretch}{3em} % prevent overfull lines
\providecommand{\tightlist}{%
  \setlength{\itemsep}{0pt}\setlength{\parskip}{0pt}}
\setcounter{secnumdepth}{5}
\usepackage{setspace}
\doublespacing
\ifLuaTeX
  \usepackage{selnolig}  % disable illegal ligatures
\fi
\IfFileExists{bookmark.sty}{\usepackage{bookmark}}{\usepackage{hyperref}}
\IfFileExists{xurl.sty}{\usepackage{xurl}}{} % add URL line breaks if available
\urlstyle{same}
\hypersetup{
  pdftitle={Project 1 Final report},
  pdfauthor={Shuai Zhu},
  hidelinks,
  pdfcreator={LaTeX via pandoc}}

\title{Project 1 Final report}
\author{Shuai Zhu}
\date{2024-10-17}

\begin{document}
\maketitle

\hypertarget{introduction}{%
\section{Introduction}\label{introduction}}

To understand the effect of physical activity on 7-year all-cause
mortality, the proposed grant application outlines a randomized trial to
investigate a novel intervention called ``ACTUP.'' This intervention is
designed to increase physical activity by a fixed 30\%
(individual-specific) among sedentary older adults. Participants will be
asked to wear wrist-worn accelerometers for 7 days to objectively
measure their physical activity levels. Activity will be evaluated using
the Total Monitor Independent Movement Summary (TMIMS), calculated as
the mean value over the 7-day period.

The study has two primary aims. First Aim is to determine whether the
ACTUP intervention leads to a reduction in the risk of 7-year all-cause
mortality (the primary endpoint) in sedentary adults aged 60-75 at the
group-average level. The second Aim is to explore whether the efficacy
of the ACTUP intervention is moderated by gender, assessing if there are
gender-specific differences in the treatment effect.

\hypertarget{method}{%
\section{Method}\label{method}}

\hypertarget{data-cleaning}{%
\subsection{data cleaning}\label{data-cleaning}}

Data were filtered to include participants aged between 60 and 75.
Sedentary status was defined as being below the 25th percentile of the
total MIMS distribution. Participants with a follow-up time of less than
7 years were excluded from the analysis. For this analysis, individuals
who died after 7 years from the start of the study were considered
alive.

\hypertarget{data-analysis}{%
\subsection{data analysis}\label{data-analysis}}

For Aim 1, we will test whether the ACTUP intervention leads to a
reduction in the risk of 7-year all-cause mortality in sedentary adults
aged 60-75 using a logistic regression model. The binary outcome will be
whether the individual died within the 7-year follow-up period, and the
primary predictor will be the treatment group (ACTUP vs.~control).

To simulate differences in mortality, the coefficient for gender from a
logistic regression model using the NHANES dataset will be utilized.
This model will have mortality as the outcome and gender as the
predictor. The simulated data will include sample sizes ranging from 100
to 2000, increasing in increments of 100. Each simulated sample will
maintain a 1:1 ratio of men to women and a 1:1 ratio of control to
treatment groups. The simulated data will then be fitted to a multiple
variable logistic regression model, with mortality as the outcome and
treatment and gender as predictors. This process will be repeated 10,000
times, and the power will be calculated as the proportion of iterations
that yield a statistically significant result, divided by 10,000.

To evaluate whether the efficacy of the ACTUP intervention is moderated
by gender, a second logistic regression model will be used, with
mortality as the outcome and treatment, gender, and their interaction
term as predictors. This model will help assess if there are significant
gender-specific differences in treatment effect.

For both aims, we will estimate the sample size required to achieve 80\%
statistical power at a significance level of 0.05.

\end{document}
