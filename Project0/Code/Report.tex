% Options for packages loaded elsewhere
\PassOptionsToPackage{unicode}{hyperref}
\PassOptionsToPackage{hyphens}{url}
%
\documentclass[
  12pt,
]{article}
\usepackage{amsmath,amssymb}
\usepackage{iftex}
\ifPDFTeX
  \usepackage[T1]{fontenc}
  \usepackage[utf8]{inputenc}
  \usepackage{textcomp} % provide euro and other symbols
\else % if luatex or xetex
  \usepackage{unicode-math} % this also loads fontspec
  \defaultfontfeatures{Scale=MatchLowercase}
  \defaultfontfeatures[\rmfamily]{Ligatures=TeX,Scale=1}
\fi
\usepackage{lmodern}
\ifPDFTeX\else
  % xetex/luatex font selection
\fi
% Use upquote if available, for straight quotes in verbatim environments
\IfFileExists{upquote.sty}{\usepackage{upquote}}{}
\IfFileExists{microtype.sty}{% use microtype if available
  \usepackage[]{microtype}
  \UseMicrotypeSet[protrusion]{basicmath} % disable protrusion for tt fonts
}{}
\makeatletter
\@ifundefined{KOMAClassName}{% if non-KOMA class
  \IfFileExists{parskip.sty}{%
    \usepackage{parskip}
  }{% else
    \setlength{\parindent}{0pt}
    \setlength{\parskip}{6pt plus 2pt minus 1pt}}
}{% if KOMA class
  \KOMAoptions{parskip=half}}
\makeatother
\usepackage{xcolor}
\usepackage[margin=1in]{geometry}
\usepackage{longtable,booktabs,array}
\usepackage{calc} % for calculating minipage widths
% Correct order of tables after \paragraph or \subparagraph
\usepackage{etoolbox}
\makeatletter
\patchcmd\longtable{\par}{\if@noskipsec\mbox{}\fi\par}{}{}
\makeatother
% Allow footnotes in longtable head/foot
\IfFileExists{footnotehyper.sty}{\usepackage{footnotehyper}}{\usepackage{footnote}}
\makesavenoteenv{longtable}
\usepackage{graphicx}
\makeatletter
\def\maxwidth{\ifdim\Gin@nat@width>\linewidth\linewidth\else\Gin@nat@width\fi}
\def\maxheight{\ifdim\Gin@nat@height>\textheight\textheight\else\Gin@nat@height\fi}
\makeatother
% Scale images if necessary, so that they will not overflow the page
% margins by default, and it is still possible to overwrite the defaults
% using explicit options in \includegraphics[width, height, ...]{}
\setkeys{Gin}{width=\maxwidth,height=\maxheight,keepaspectratio}
% Set default figure placement to htbp
\makeatletter
\def\fps@figure{htbp}
\makeatother
\setlength{\emergencystretch}{3em} % prevent overfull lines
\providecommand{\tightlist}{%
  \setlength{\itemsep}{0pt}\setlength{\parskip}{0pt}}
\setcounter{secnumdepth}{5}
\usepackage{setspace}
\doublespacing
\usepackage{fontspec}
\setmainfont{Times New Roman}
\ifLuaTeX
  \usepackage{selnolig}  % disable illegal ligatures
\fi
\IfFileExists{bookmark.sty}{\usepackage{bookmark}}{\usepackage{hyperref}}
\IfFileExists{xurl.sty}{\usepackage{xurl}}{} % add URL line breaks if available
\urlstyle{same}
\hypersetup{
  pdftitle={Project 0 final report},
  pdfauthor={Shuai Zhu},
  hidelinks,
  pdfcreator={LaTeX via pandoc}}

\title{Project 0 final report}
\author{Shuai Zhu}
\date{2024-10-01}

\begin{document}
\maketitle

\begin{center}\rule{0.5\linewidth}{0.5pt}\end{center}

\hypertarget{introduction}{%
\section{Introduction}\label{introduction}}

The data for this project is on a set of resting state functional
connectivity(RSFC) and a set of measures of physical function. The
investigator collected 30 resting state functional connectivity and 15
physical functions of 300 participants. The objective of this report is
to outline an analysis plan aimed at addressing the following research
question:

\begin{itemize}
\tightlist
\item
  what measures of resting state functional connectivity are associated
  with individual physical function.
\end{itemize}

To achieve this, we employ a series of separate linear regression
models, one for each measure of physical function.

\hypertarget{method}{%
\section{Method}\label{method}}

\hypertarget{data-collection-and-data-cleaning}{%
\subsection{Data collection and Data
cleaning}\label{data-collection-and-data-cleaning}}

The study collected data from 300 participants. For each participant,
two sets of data were obtained. 30 measures of resting-state functional
connectivity (RSFC): These measures, labeled as X.1 to X.30, were used
as predictors in the analysis. 15 measures of physical function: These
measures, labeled Y.1 to Y.15, were used as outcomes in the study. No
formal data cleaning procedures were applied to the dataset. All
available data from the 300 participants were included in the analysis
without the removal of outliers,

\hypertarget{data-analysis}{%
\subsection{Data analysis}\label{data-analysis}}

A total of 15 multivariable linear regression models were conducted to
examine the associations between 30 resting-state functional
connectivity predictors and 15 physical function outcomes, resulting in
450 tests. 480 parameters were estimated across all models, but the
analysis focused on 450 parameters, excluding the intercept.

\hypertarget{multiple-testing-and-family-wise-error-rate-fwer}{%
\subsubsection{Multiple Testing and Family-Wise Error Rate
(FWER)}\label{multiple-testing-and-family-wise-error-rate-fwer}}

Given the large number of tests conducted, controlling for the
family-wise Type I error rate (FWER) was essential to reduce the risk of
false discoveries. The FWER represents the probability of making at
least one false positive across all 450 tests. To address this, the
Bonferroni correction was applied. By dividing the desired significance
level (0.05) by the total number of tests (450), the adjusted threshold
for significance was set to a p-value of less than 0.000111. Any p-value
below this threshold was considered statistically significant.

While the Bonferroni correction is straightforward to implement, it is
known to be overly conservative, especially when handling a large number
of tests, as it reduces statistical power by inflating the threshold for
significance. Therefore, it may lead to the exclusion of predictors that
are truly significant.

\hypertarget{alternative-correction-approach-bootstrap-based-adjustment}{%
\subsubsection{Alternative Correction Approach: Bootstrap-Based
Adjustment}\label{alternative-correction-approach-bootstrap-based-adjustment}}

An alternative bootstrap-based approach was employed to address the
limitations of the Bonferroni correction. This method leverages
resampling to estimate the empirical distribution of the regression
coefficients. Specifically, bootstrap sampling was used to generate
distributions for the estimated beta coefficients, which were then used
to compute 95\% confidence intervals. The bootstrap method involved
repeatedly resampling the data with replacement and refitting the models
to generate a distribution of coefficient estimates for each predictor.

Predictors with confidence intervals that did not include zero were
considered statistically significant. Unlike the Bonferroni method, the
bootstrap-based approach does not rely on p-values and provides a
measure of significance by directly assessing the variability of the
coefficients. The results of this method, including the significant
predictors and their corresponding confidence intervals, are presented
in Table 1.

All analyses were conducted using R version 4.3.3.

\hypertarget{results}{%
\section{Results}\label{results}}

Fifteen multivariable regression models were fitted to identify
associations between 30 resting-state functional connectivity predictors
and each of the 15 physical function outcomes. The significance of each
predictor across the models is presented in Table 2 (Appendix). Without
adjustment, 79 predictors were significantly associated with the
physical function outcomes. Among these, predictor X.5 was significant
for 9 outcomes.

Applying the Bonferroni correction reduced the number of significant
predictors to 6 (Table 3). As mentioned in the Methods section, the
Bonferroni correction was overly conservative. An alternative
bootstrap-based approach increased the number of significant predictors
to 44 (Appendix Table 3). P-values were not calculated with this method;
instead, a 95\% confidence interval that did not include 0 was
considered significant.

\hypertarget{conclusions}{%
\section{Conclusions}\label{conclusions}}

Based on the results of the bootstrap-based adjustment, predictor X.5
was found to be significantly associated with most physical function
outcomes, suggesting a potential relationship between this resting-state
functional connectivity measure and physical function. For example, the
relationship between X.5 and Y.1 can be interpreted as follows: for each
one-unit increase in resting-state functional connectivity (X.5), the
physical function Y.1 increases by 1.31 units, with a 95\% confidence
interval ranging from 0.0657 to 2.55. This indicates that the effect of
X.5 on Y.1 is likely positive, and the association is statistically
significant, as the confidence interval does not include zero. One
advantage of the bootstrap method is its ability to control the
family-wise error rate (FWER) more effectively compared to the
Bonferroni correction. This approach is less conservative than
Bonferroni's, allowing for greater statistical power while still
controlling the FWER.

\begin{longtable}[]{@{}
  >{\raggedleft\arraybackslash}p{(\columnwidth - 8\tabcolsep) * \real{0.1026}}
  >{\raggedleft\arraybackslash}p{(\columnwidth - 8\tabcolsep) * \real{0.1410}}
  >{\raggedleft\arraybackslash}p{(\columnwidth - 8\tabcolsep) * \real{0.1154}}
  >{\raggedleft\arraybackslash}p{(\columnwidth - 8\tabcolsep) * \real{0.3205}}
  >{\raggedleft\arraybackslash}p{(\columnwidth - 8\tabcolsep) * \real{0.3205}}@{}}
\caption{Significant Predictors after Adjustment of
bootstrap}\tabularnewline
\toprule\noalign{}
\begin{minipage}[b]{\linewidth}\raggedleft
Outcome
\end{minipage} & \begin{minipage}[b]{\linewidth}\raggedleft
Predicator
\end{minipage} & \begin{minipage}[b]{\linewidth}\raggedleft
Estimate
\end{minipage} & \begin{minipage}[b]{\linewidth}\raggedleft
Lower confident interval
\end{minipage} & \begin{minipage}[b]{\linewidth}\raggedleft
Upper confident interval
\end{minipage} \\
\midrule\noalign{}
\endfirsthead
\toprule\noalign{}
\begin{minipage}[b]{\linewidth}\raggedleft
Outcome
\end{minipage} & \begin{minipage}[b]{\linewidth}\raggedleft
Predicator
\end{minipage} & \begin{minipage}[b]{\linewidth}\raggedleft
Estimate
\end{minipage} & \begin{minipage}[b]{\linewidth}\raggedleft
Lower confident interval
\end{minipage} & \begin{minipage}[b]{\linewidth}\raggedleft
Upper confident interval
\end{minipage} \\
\midrule\noalign{}
\endhead
\bottomrule\noalign{}
\endlastfoot
1 & 5 & 1.307 & 0.066 & 2.549 \\
2 & 1 & -1.017 & -2.023 & -0.012 \\
2 & 5 & 1.823 & 0.565 & 3.081 \\
2 & 17 & -1.714 & -3.384 & -0.043 \\
2 & 23 & 0.947 & 0.439 & 1.456 \\
3 & 26 & 0.829 & 0.059 & 1.599 \\
4 & 5 & 1.825 & 0.555 & 3.094 \\
4 & 29 & -1.600 & -2.878 & -0.322 \\
5 & 7 & -0.934 & -1.823 & -0.045 \\
5 & 23 & 0.827 & 0.290 & 1.364 \\
6 & 5 & 1.405 & 0.104 & 2.706 \\
6 & 7 & -0.882 & -1.753 & -0.011 \\
6 & 20 & -2.069 & -3.928 & -0.211 \\
6 & 29 & -1.756 & -3.065 & -0.447 \\
7 & 15 & 1.772 & 0.237 & 3.307 \\
7 & 23 & 0.837 & 0.314 & 1.360 \\
7 & 27 & -1.805 & -3.505 & -0.104 \\
7 & 29 & -1.686 & -2.988 & -0.384 \\
8 & 4 & -1.253 & -2.330 & -0.176 \\
8 & 5 & 1.654 & 0.348 & 2.960 \\
8 & 17 & -1.796 & -3.530 & -0.062 \\
8 & 23 & 1.157 & 0.629 & 1.684 \\
8 & 26 & 0.997 & 0.215 & 1.780 \\
8 & 27 & -1.953 & -3.669 & -0.236 \\
9 & 7 & -1.016 & -1.862 & -0.171 \\
9 & 23 & 0.609 & 0.098 & 1.119 \\
10 & 5 & 1.407 & 0.143 & 2.671 \\
10 & 7 & -0.965 & -1.811 & -0.119 \\
10 & 26 & 1.082 & 0.325 & 1.839 \\
11 & 4 & -1.253 & -2.318 & -0.188 \\
11 & 26 & 1.410 & 0.636 & 2.184 \\
11 & 27 & -1.821 & -3.519 & -0.123 \\
11 & 29 & -1.561 & -2.860 & -0.261 \\
12 & 4 & -1.168 & -2.278 & -0.059 \\
12 & 5 & 1.434 & 0.089 & 2.779 \\
12 & 15 & 1.686 & 0.090 & 3.282 \\
12 & 23 & 0.680 & 0.137 & 1.224 \\
13 & 5 & 1.592 & 0.228 & 2.956 \\
13 & 15 & 1.830 & 0.211 & 3.449 \\
13 & 22 & 1.424 & 0.054 & 2.794 \\
14 & 23 & 1.004 & 0.452 & 1.556 \\
15 & 1 & -1.129 & -2.193 & -0.066 \\
15 & 22 & 1.395 & 0.059 & 2.731 \\
15 & 29 & -1.663 & -3.001 & -0.324 \\
\end{longtable}

\newpage

\hypertarget{appendix}{%
\section{Appendix}\label{appendix}}

\begin{longtable}[]{@{}
  >{\raggedright\arraybackslash}p{(\columnwidth - 8\tabcolsep) * \real{0.1026}}
  >{\raggedright\arraybackslash}p{(\columnwidth - 8\tabcolsep) * \real{0.1410}}
  >{\raggedleft\arraybackslash}p{(\columnwidth - 8\tabcolsep) * \real{0.1154}}
  >{\raggedleft\arraybackslash}p{(\columnwidth - 8\tabcolsep) * \real{0.3205}}
  >{\raggedleft\arraybackslash}p{(\columnwidth - 8\tabcolsep) * \real{0.3205}}@{}}
\caption{Significant Predictors without adjustement}\tabularnewline
\toprule\noalign{}
\begin{minipage}[b]{\linewidth}\raggedright
Outcome
\end{minipage} & \begin{minipage}[b]{\linewidth}\raggedright
Predicator
\end{minipage} & \begin{minipage}[b]{\linewidth}\raggedleft
Estimate
\end{minipage} & \begin{minipage}[b]{\linewidth}\raggedleft
Lower confident interval
\end{minipage} & \begin{minipage}[b]{\linewidth}\raggedleft
Upper confident interval
\end{minipage} \\
\midrule\noalign{}
\endfirsthead
\toprule\noalign{}
\begin{minipage}[b]{\linewidth}\raggedright
Outcome
\end{minipage} & \begin{minipage}[b]{\linewidth}\raggedright
Predicator
\end{minipage} & \begin{minipage}[b]{\linewidth}\raggedleft
Estimate
\end{minipage} & \begin{minipage}[b]{\linewidth}\raggedleft
Lower confident interval
\end{minipage} & \begin{minipage}[b]{\linewidth}\raggedleft
Upper confident interval
\end{minipage} \\
\midrule\noalign{}
\endhead
\bottomrule\noalign{}
\endlastfoot
Y.1 & X.1 & -0.867 & -1.612 & -1.612 \\
Y.1 & X.5 & 1.307 & 0.375 & 0.375 \\
Y.1 & X.9 & -1.739 & -3.334 & -3.334 \\
Y.1 & X.15 & 1.284 & 0.179 & 0.179 \\
Y.1 & X.17 & -1.334 & -2.571 & -2.571 \\
Y.2 & X.1 & -1.017 & -1.772 & -1.772 \\
Y.2 & X.5 & 1.823 & 0.879 & 0.879 \\
Y.2 & X.7 & -0.732 & -1.364 & -1.364 \\
Y.2 & X.17 & -1.714 & -2.967 & -2.967 \\
Y.2 & X.23 & 0.947 & 0.566 & 0.566 \\
Y.3 & X.1 & -0.890 & -1.662 & -1.662 \\
Y.3 & X.3 & 2.427 & 0.517 & 0.517 \\
Y.3 & X.9 & -2.033 & -3.684 & -3.684 \\
Y.3 & X.20 & -1.615 & -2.992 & -2.992 \\
Y.3 & X.26 & 0.829 & 0.251 & 0.251 \\
Y.4 & X.5 & 1.825 & 0.872 & 0.872 \\
Y.4 & X.21 & 2.729 & 0.091 & 0.091 \\
Y.4 & X.29 & -1.600 & -2.559 & -2.559 \\
Y.5 & X.7 & -0.934 & -1.601 & -1.601 \\
Y.5 & X.15 & 1.479 & 0.296 & 0.296 \\
Y.5 & X.20 & -1.690 & -3.114 & -3.114 \\
Y.5 & X.23 & 0.827 & 0.424 & 0.424 \\
Y.5 & X.28 & 2.894 & 0.101 & 0.101 \\
Y.6 & X.5 & 1.405 & 0.428 & 0.428 \\
Y.6 & X.7 & -0.882 & -1.536 & -1.536 \\
Y.6 & X.9 & -1.733 & -3.405 & -3.405 \\
Y.6 & X.19 & 3.338 & 0.139 & 0.139 \\
Y.6 & X.20 & -2.069 & -3.464 & -3.464 \\
Y.6 & X.21 & 2.781 & 0.078 & 0.078 \\
Y.6 & X.23 & 0.464 & 0.069 & 0.069 \\
Y.6 & X.28 & 2.810 & 0.075 & 0.075 \\
Y.6 & X.29 & -1.756 & -2.738 & -2.738 \\
Y.7 & X.1 & -0.783 & -1.559 & -1.559 \\
Y.7 & X.5 & 0.994 & 0.023 & 0.023 \\
Y.7 & X.12 & -1.229 & -2.435 & -2.435 \\
Y.7 & X.13 & -2.461 & -4.888 & -4.888 \\
Y.7 & X.15 & 1.772 & 0.620 & 0.620 \\
Y.7 & X.18 & -2.099 & -4.123 & -4.123 \\
Y.7 & X.23 & 0.837 & 0.444 & 0.444 \\
Y.7 & X.26 & 0.706 & 0.124 & 0.124 \\
Y.7 & X.27 & -1.805 & -3.081 & -3.081 \\
Y.7 & X.29 & -1.686 & -2.663 & -2.663 \\
Y.8 & X.4 & -1.253 & -2.062 & -2.062 \\
Y.8 & X.5 & 1.654 & 0.674 & 0.674 \\
Y.8 & X.17 & -1.796 & -3.098 & -3.098 \\
Y.8 & X.20 & -1.565 & -2.965 & -2.965 \\
Y.8 & X.23 & 1.157 & 0.760 & 0.760 \\
Y.8 & X.26 & 0.997 & 0.410 & 0.410 \\
Y.8 & X.27 & -1.953 & -3.241 & -3.241 \\
Y.9 & X.7 & -1.016 & -1.651 & -1.651 \\
Y.9 & X.23 & 0.609 & 0.226 & 0.226 \\
Y.10 & X.1 & -0.876 & -1.634 & -1.634 \\
Y.10 & X.5 & 1.407 & 0.459 & 0.459 \\
Y.10 & X.7 & -0.965 & -1.600 & -1.600 \\
Y.10 & X.9 & -1.945 & -3.569 & -3.569 \\
Y.10 & X.26 & 1.082 & 0.514 & 0.514 \\
Y.11 & X.3 & 2.276 & 0.357 & 0.357 \\
Y.11 & X.4 & -1.253 & -2.053 & -2.053 \\
Y.11 & X.17 & -1.508 & -2.795 & -2.795 \\
Y.11 & X.26 & 1.410 & 0.829 & 0.829 \\
Y.11 & X.27 & -1.821 & -3.095 & -3.095 \\
Y.11 & X.29 & -1.561 & -2.536 & -2.536 \\
Y.12 & X.4 & -1.168 & -2.001 & -2.001 \\
Y.12 & X.5 & 1.434 & 0.424 & 0.424 \\
Y.12 & X.15 & 1.686 & 0.488 & 0.488 \\
Y.12 & X.22 & 1.283 & 0.269 & 0.269 \\
Y.12 & X.23 & 0.680 & 0.272 & 0.272 \\
Y.12 & X.26 & 0.731 & 0.126 & 0.126 \\
Y.13 & X.3 & 2.087 & 0.060 & 0.060 \\
Y.13 & X.5 & 1.592 & 0.568 & 0.568 \\
Y.13 & X.15 & 1.830 & 0.615 & 0.615 \\
Y.13 & X.22 & 1.424 & 0.396 & 0.396 \\
Y.14 & X.23 & 1.004 & 0.590 & 0.590 \\
Y.15 & X.1 & -1.129 & -1.928 & -1.928 \\
Y.15 & X.18 & -2.280 & -4.361 & -4.361 \\
Y.15 & X.19 & 3.788 & 0.516 & 0.516 \\
Y.15 & X.22 & 1.395 & 0.392 & 0.392 \\
Y.15 & X.25 & -2.367 & -4.347 & -4.347 \\
Y.15 & X.29 & -1.663 & -2.667 & -2.667 \\
\end{longtable}

\begin{longtable}[]{@{}
  >{\raggedright\arraybackslash}p{(\columnwidth - 8\tabcolsep) * \real{0.1026}}
  >{\raggedright\arraybackslash}p{(\columnwidth - 8\tabcolsep) * \real{0.1410}}
  >{\raggedleft\arraybackslash}p{(\columnwidth - 8\tabcolsep) * \real{0.1154}}
  >{\raggedleft\arraybackslash}p{(\columnwidth - 8\tabcolsep) * \real{0.3205}}
  >{\raggedleft\arraybackslash}p{(\columnwidth - 8\tabcolsep) * \real{0.3205}}@{}}
\caption{Significant Predictors with Bonferroni
correlation}\tabularnewline
\toprule\noalign{}
\begin{minipage}[b]{\linewidth}\raggedright
Outcome
\end{minipage} & \begin{minipage}[b]{\linewidth}\raggedright
Predicator
\end{minipage} & \begin{minipage}[b]{\linewidth}\raggedleft
Estimate
\end{minipage} & \begin{minipage}[b]{\linewidth}\raggedleft
Lower confident interval
\end{minipage} & \begin{minipage}[b]{\linewidth}\raggedleft
Upper confident interval
\end{minipage} \\
\midrule\noalign{}
\endfirsthead
\toprule\noalign{}
\begin{minipage}[b]{\linewidth}\raggedright
Outcome
\end{minipage} & \begin{minipage}[b]{\linewidth}\raggedright
Predicator
\end{minipage} & \begin{minipage}[b]{\linewidth}\raggedleft
Estimate
\end{minipage} & \begin{minipage}[b]{\linewidth}\raggedleft
Lower confident interval
\end{minipage} & \begin{minipage}[b]{\linewidth}\raggedleft
Upper confident interval
\end{minipage} \\
\midrule\noalign{}
\endhead
\bottomrule\noalign{}
\endlastfoot
Y.2 & X.23 & 0.947 & 0.566 & 0.566 \\
Y.5 & X.23 & 0.827 & 0.424 & 0.424 \\
Y.7 & X.23 & 0.837 & 0.444 & 0.444 \\
Y.8 & X.23 & 1.157 & 0.760 & 0.760 \\
Y.11 & X.26 & 1.410 & 0.829 & 0.829 \\
Y.14 & X.23 & 1.004 & 0.590 & 0.590 \\
\end{longtable}

\end{document}
